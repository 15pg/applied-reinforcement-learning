\documentclass[11pt]{article} % For LaTeX2e
\usepackage{rldmsubmit,palatino}
\usepackage{graphicx}

\title{A Complementary Way of Teaching Reinforcement Learning and Decision Making}


\author{
Miguel Morales\thanks{http://www.mimoralea.com} \\
Department of Computer Science \\
Georgia Institute of Technology \\
Atlanta, GA 30332 \\
\texttt{mimoralea@gatech.edu} \\
}

% The \author macro works with any number of authors. There are two commands
% used to separate the names and addresses of multiple authors: \And and \AND.
%
% Using \And between authors leaves it to \LaTeX{} to determine where to break
% the lines. Using \AND forces a linebreak at that point. So, if \LaTeX{}
% puts 3 of 4 authors names on the first line, and the last on the second
% line, try using \AND instead of \And before the third author name.

\newcommand{\fix}{\marginpar{FIX}}
\newcommand{\new}{\marginpar{NEW}}

\begin{document}

\maketitle

\begin{abstract}
  Reinforcement Learning and Decision Making is a complex subject. Being the
  focus of research of a variety of fields including artificial intelligence,
  psychology, machine learning, operations research, control theory, animal
  and human neuroscience, economics, and ethology, it is expected that the
  vast amount of available information could become counterproductive.
  Beginners often find themselves lost while trying to grasp the key concepts
  that are truly vital for understanding. Additionally, reinforcement learning
  and decision making, being a relatively new field, is often taught by
  world-class researchers that frequently unintentionally omit explaining
  core concepts that might seem too basic, but are as well fundamental. This
  creates a gap of knowledge that, if left unfilled, causes trouble for learning
  the more advanced topics.

  Fortunately, as reinforcement learning and decision making is also studied
  by fields like animal and human neuroscience, ethology, and psychology, often
  the concepts can be taught on an intuitive level. The notion of learning
  by interacting with the environment should be easy to understand to all of
  us as this is one of the ways we learn. The work described on this paper
  is an attempt to deliver reinforcement learning and decision making concepts
  using different teaching techniques that potentially promote intuitive learning.
  The idea is that these work would serve beginners fill the gaps of knowledge,
  as a `primer` to prepare them for the more advanced and in-depth material.
\end{abstract}

\keywords{
teaching tutorials jupyter intuition hands-on
}

\acknowledgements{
  I am thankful to my mentor, Kenneth Brooks, for providing assistance when
  navigating the field of Educational Technology. Also, when giving direct,
  concise and clear feedback on how to make this project better. Thank you
  to all my peers who also provided sincere feedback throughout the semester.
  I hope to see you all enjoying our OMSCS course in Reinforcement Learning
  and Decision Making. It is a rewarding experience. Pun intended.}

\startmain % to start the main 1-4 pages of the submission.

\section{Introduction}

RLDM requires electronic submissions.  This year's electronic
submission site is   
\begin{center}
   https://cmt3.research.microsoft.com/RLDM2017/
\end{center}

Please read the instructions below, and follow them faithfully. Note
that there is also a template \verb+rldm.rtf+ for Microsoft Word,
which is available from the website below.
\subsection{Style}

Papers to be submitted to RLDM must be prepared according to the
instructions presented here. Papers consist of a \emph{title}, which
has a maximum of 100 characters, an \emph{abstract}, which is a
maximum of 2000 characters, up to five key words, and an
\emph{extended abstract}, which starts on the second page, and can be
between one and four pages. Figures and references should be included
in the latter.

Authors preferring \LaTeX{} are requested to use the RLDM \LaTeX{}
style files obtainable at the RLDM website at
\begin{center}
   http://www.rldm.org/
\end{center}
The file \verb+rldm.pdf+ contains these instructions and illustrates
the various formatting requirements your RLDM paper must
satisfy. There is a \LaTeX{} style file called \verb+rldmsubmit.sty+,
and a \LaTeX{} file \verb+rldm.tex+, which may be used as a ``shell''
for writing your paper. All you have to do is replace the author,
title, abstract, keywords, acknowledgements and text of the paper with
your own. The file
\verb+rldm.rtf+ is provided as an equivalent shell for Microsoft Word users. 

\section{Background and Definitions}
\label{back_def}

The paper size for RLDM is ``US Letter'' (rather than ``A4''). Margins
are 1.5cm around all sides. Use 11~point type with a vertical spacing
of 12~points. Palatino is the preferred typeface throughout.
Paragraphs are separated by 1/2~line space, with no indentation.

Paper title is 17~point, initial caps/lower case, bold, centered between
2~horizontal rules. Top rule is 4~points thick and bottom rule is 1~point
thick. Allow 0.6cm space above and below title to rules. 

The lead author's name is to be listed first (left-most), and
the co-authors' names (if different address) are set to follow. If
there is only one co-author, list both author and co-author side by side.

\section{Using Intuition as a Primer}

Please prepare PostScript or PDF files with paper size ``US Letter''.
The -t letter option on dvips will produce US Letter files.

This document is an example of \texttt{thebibliography} environment using 
in bibliography management. Three items are cited: \textit{The \LaTeX\ Companion} 
book \cite{latexcompanion}, the Einstein journal paper \cite{einstein}, and the 
Donald Knuth's website \cite{knuthwebsite}. The \LaTeX\ related items are
\cite{latexcompanion,knuthwebsite}.

\section{Creating Awareness through Experimentation}

Please prepare PostScript or PDF files with paper size ``US Letter''.
The -t letter option on dvips will produce US Letter files.

\section{Building Knowledge through Assigned Readings}

Please prepare PostScript or PDF files with paper size ``US Letter''.
The -t letter option on dvips will produce US Letter files.

This document is an example of \texttt{thebibliography} environment using 
in bibliography management. Three items are cited: \textit{The \LaTeX\ Companion} 
book \cite{latexcompanion}, the Einstein journal paper \cite{einstein}, and the 
Donald Knuth's website \cite{knuthwebsite}. The \LaTeX\ related items are
\cite{latexcompanion,knuthwebsite}.

\medskip
 
\begin{thebibliography}{9}
\bibitem{latexcompanion} 
Michel Goossens, Frank Mittelbach, and Alexander Samarin. 
\textit{The \LaTeX\ Companion}. 
Addison-Wesley, Reading, Massachusetts, 1993.
 
\bibitem{einstein} 
Albert Einstein. 
\textit{Zur Elektrodynamik bewegter K{\"o}rper}. (German) 
[\textit{On the electrodynamics of moving bodies}]. 
Annalen der Physik, 322(10):891–921, 1905.
 
\bibitem{knuthwebsite} 
Knuth: Computers and Typesetting,
\\\texttt{http://www-cs-faculty.stanford.edu/\~{}uno/abcde.html}
\end{thebibliography}


\end{document}
